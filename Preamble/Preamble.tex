% Encoding set and preamble settings
\usepackage[utf8]{inputenc}
% Settings - udkommenter for at slå fra
\newcommand*{\draft}{} % Skal der compiles som draft skal slåes fra når projekt aflevers.
\newcommand*{\CMDCode}{} % Kode kommandoer
\newcommand*{\CMDFixMe}{} % Aktiver fixme

% Sæt dokumentinformationerne
\newcommand{\rapportnavn}{Somthing about giraf}
\newcommand{\rapportsubtitle}{?????????????????????????}
\newcommand{\supervisor}{Ulrik Mathias Nyman}
\newcommand{\gruppen}{sw614f16}
\newcommand{\gruppenummer}{sw614f16}
\newcommand{\afldate}{Maj 2016}

% Ekstra kommandoer
\newcommand{\figuregroup}{}
\newcommand{\tabelgroup}{}
\newcommand{\codegroup}{}
\newcommand{\screenshotgroup}{}

% Definer forkortelser
%\let\newfootnote\footnote % Kopier footnote
%\newcommand{\fn}[2]{\newfootnote{#1: #2}}
%\renewcommand{\footnote}{\errmessage{Brug fn i stedet for footnote}}

\let\newfootnote\footnote % Kopier footnote
\newcommand{\fn}[1]{\newfootnote{#1}}
\renewcommand{\footnote}{\errmessage{Brug fn i stedet for footnote}}


%TH
%${th}^$
\newcommand\nd{\textsuperscript{nd}\xspace}
\newcommand\rd{\textsuperscript{rd}\xspace}
\newcommand\nth{\textsuperscript{th}\xspace}



% Packages
\usepackage{array}% http://ctan.org/pkg/array
\usepackage{parskip} % Creates extra spacing between paragraphs
\usepackage{tabularx} % For creating tables
\usepackage{multirow} % Tables with multiple rows
\usepackage{threeparttable} % For footnotes in tables
\usepackage{fourier} % Default font
\usepackage[english]{babel} % Report language
\usepackage{geometry} % Report margins
%\usepackage[hcentering, bindingoffset=4mm]{geometry} % Report margins
\usepackage{fancyhdr} % Pretty headers and footers
\usepackage{lastpage} % For finding lastpage
\usepackage{hyperref} % clickable references for labels in PDF
\usepackage{float} % For constructing figures properly
\usepackage{wrapfig} % allows figures to wrap text -http://ctan.org/pkg/wrapfig
\usepackage{url} % For making URL
\usepackage{amssymb,amsmath} % For making large math expressions
\usepackage{titlesec} % For changing section sizes
\usepackage{xr} % Makes it possible to cross reference in multiple files
\usepackage{amsmath} % For making matematical expressions
\usepackage{mathtools} % For making matematical expressions
\usepackage[toc,page]{appendix} % Appendix
\usepackage{enumitem} % Enumerated lists
\usepackage[boxruled,linesnumbered,noend]{algorithm2e} % Algorithms
\usepackage{graphicx} % Images
\usepackage{pdfpages} % PDF pages
\usepackage{color} % For using colors
\usepackage{soulutf8} % For highlighting
\usepackage{multicol} % For itemizing with multiple columns
\usepackage{cleveref} % Enhances LATEX's cross-referencing features
\usepackage[nounderscore]{syntax} % Syntax for Context Free Grammar
\usepackage{listings} % For writing code.
\usepackage{csquotes} % Used for making quotes
\usepackage[backend=biber, style=ieee]{biblatex} % Litterature list
\usepackage{todonotes} %adds todo list
\usepackage{blindtext} % adds some test without meanning for a todo


\usepackage{xspace}

\setlength\extrarowheight{4pt} % Extra spacing in tables between row elements

% Draft settings
\ifdefined\draft\usepackage{Preamble/Packages/currfile}\fi % Show file names
\ifdefined\draft\usepackage[inline]{showlabels}\fi % Show label names

% Insert litterature
\addbibresource{Chapters/Kilder.bib}

% Headlines for algorithms
% http://tex.stackexchange.com/questions/41012/change-the-algorithm-enumeration-in-the-algorithm2e-package
\renewcommand{\listalgorithmcfname}{Algorithms}%
\renewcommand{\algorithmcfname}{Algorithm}%
\renewcommand{\algorithmautorefname}{algorithm}%
\renewcommand{\algorithmcflinename}{line}%
\renewcommand{\procedureautorefname}{procedure}%

% Reference for listings
\crefname{lstlisting}{listing}{listings}

% https://tex.stackexchange.com/questions/129400/multiple-letters-without-spacing-in-math
\newcommand{\mli}[1]{\mathit{#1}}

% Dokument informationskommandoer
\newcommand{\currentpage}{\thepage}     % nuværende side
\newcommand{\numpages}{\pageref{LastPage}}% antal sider
\newcommand{\sidetal}{Page {\currentpage} of {\numpages}} % side informationer
\newcommand{\pagetitle}{\rightmark}     % nuværende sektion

% Report meta information
\title{\rapportnavn}
\author{\gruppen}
\date{\afldate}

% Setup af header og footer
\fancypagestyle{plain}{
    % Nulstil header og footer
    \fancyhead{}
    \fancyfoot{}

    % Setup header
    % Left / Right - Even / Odd
    \fancyhead[LE]{\rapportnavn}        % Lige sider
    \fancyhead[RO]{\pagetitle}          % Ulige sider

    % Sæt footer op
    \fancyfoot[LE]{\sidetal}            % Lige sider
    \fancyfoot[RO]{\sidetal}            % Ulige sider
    
    % Sæt kun filnavn på en draft
    \ifdefined\draft
        \fancyfoot[LO]{\currfilepath}   % Ulige sider
        \fancyfoot[RE]{\currfilepath}   % Lige sider
    \fi
}

\setlength{\headheight}{15pt}

% Sætter pagestyle til plain - header og footer
\pagestyle{plain}

% Fjerner ligegyldig whitespace - men rykker ting op så vi ikke har teksten på slut
\raggedbottom

% Fjerner punktummet efter sektion-nummer i header (2.2. XXX --> 2.2 XXX)
\renewcommand\sectionmark[1]{%
  \markright{\MakeUppercase{\textbf{\thesection}\ #1}}}

% Fjerner afstand mellem kapitel og sætter størrelsen på skriften - se evt. her: http://tex.stackexchange.com/questios/63390/how-to-decrease-spacing-before-chapter-title
\titleformat{\chapter}[display]{\normalfont\huge\bfseries}{\chaptertitlename \hspace{0.1cm} \thechapter}{20pt}{\Huge}
\titlespacing*{\chapter}{0pt}{0pt}{20pt}

% Indent til eller fra
\newcommand{\enableIndent}{\setlength\parindent{12pt}}
\newcommand{\disableIndent}{\setlength\parindent{0pt}}
\disableIndent 

% Opretter subsubsubsection og subsubsubsubsection
\newcommand{\subsubsubsection}[1]{\noindent\paragraph{#1}\mbox{}\\}
\newcommand{\subsubsubsubsection}[1]{\noindent\subparagraph{#1}\mbox{}\\}

% Skriftstørrelse på sections
\titleformat*{\section}{\LARGE\bfseries}
\titleformat*{\subsection}{\Large\bfseries}
\titleformat*{\subsubsection}{\large\bfseries}
\titleformat*{\paragraph}{\normalsize\bfseries}
\titleformat*{\subparagraph}{\normalsize\bfseries}

% Definerer antallet af overskrifter der har tal, samt antallet af overskrifter i indholdsfortegnelsen
\setcounter{secnumdepth}{2}             % Antallet af overskrifter der har et nr - dybden
\setcounter{tocdepth}{1}                % Antallet af overskrifter i indholdsfortegnelse - dybden - eks. 3.2

% Orddeling
\hyphenation{}

% Ændrer indstillingerne for caption for figurer, tabeller og kodestykker - tilføjer bla. kusiv og gør typenavnet bold (eks. Figur 1.1: bliver fed)
\usepackage[font=small,format=plain,labelfont=bf,up,textfont=it,up]{caption}

% FixMe
\ifdefined\CMDFixMe
    \begin{titlepage}
    \setlength{\textwidth}{15cm}
	\noindent
	\begin{nopagebreak}
	{\samepage 
			\begin{tabular}{lr}
				\parbox{0.5\textwidth}{\raisebox{11mm}
					{\includegraphics[height=3.0cm]{Images/aauLogo.pdf}}
				} &
				\parbox{0.5\textwidth}{
					\small
					\begin{tabular}{l}
						{\sf\small \textbf{Cassiopeia}}\\
						{\sf\small \textbf{House of Computer Science}} \\
						{\sf\small Selma Lagerlöfsvej 300} \\
						{\sf\small 9220 Aalborg Ø} \\
						{\sf\small Tlf. 9940 9940} \\
						{\sf\small \url{http://www.cs.aau.dk/}}
					\end{tabular}
				}
			\end{tabular}
			
			\noindent
			\begin{tabular}{cc}
				\parbox{7cm}{
					\begin{description}
			
						\item {\bf Title:} 
						
							\textbf{\rapportnavn}\\
			            
			  
						\item {\bf Theme:}
						
							Developing Complex Software Systems
		                \newline
						\item {\bf Project Period:}\\
			  				P6, Spring 2016
			 				%\hspace{4cm}
			 			\newline
						\item {\bf Project Group:}\\
							\gruppenummer
			  				\hspace{4cm}
			  			\newline
						\item {\bf Participants:}\\
							Maritn Haaber\\
                            Kenneth Haunstrup\\
                            Kenneth Stick\\
                            Thomas Bjelbo Thomsen\\
							\hspace{2cm}
						
						\item {\bf Supervisor:}\\
							\supervisor
						\newline

						\item {\bf Copies:} write at end of project %https://www.moodle.aau.dk/course/view.php?id=11232
						\newline
						\item {\bf Number of pages:} \numpages
						\newline
						\item {\bf Appendix:} \ref{FIRSTAPPENDIX} - \ref{LASTAPPENDIX}
						\newline
						%\item {\bf Bilagsantal og art:} 1x CD
						\item {\bf Date of Completion:} \today
					\end{description}
					\vfill
				} &
				\parbox{8cm}{
					\vspace{.15cm}
					\hfill 
					\begin{tabular}{l}
						{\bf Abstract:}\bigskip \\
						\fbox{
							\parbox{6.5cm}{\smallskip
								{\vfill{\small \input{Chapters/Initial/TitlePage/Synopsis.tex}
								\smallskip}}
							}
						}
  					\end{tabular}
  				}
			\end{tabular}
		}
	\end{nopagebreak}
\end{titlepage}
\clearpage
\newcommand{\doublesignature}[4]{
    \noindent
    \begin{minipage}[b]{0.45\textwidth}
        \centering
        \rule{\textwidth}{0.5pt}\\
        #1\\
        {\footnotesize <#2@student.aau.dk>}
    \end{minipage}
    \hfill
    \begin{minipage}[b]{0.45\textwidth}
        \centering
        \rule{\textwidth}{0.5pt}\\
        #3\\
        {\footnotesize <#4@student.aau.dk>}
        \end{minipage}
    \vspace{4\baselineskip}
% Very important whitespace

}

\begin{titlepage}
    \vspace*{\fill}
    \doublesignature{Maritn Haaber}{mhaabe13}{Kenneth Haunstrup}{??}
    \doublesignature{Kenneth Stick}{????}{Thomas Bjelbo Thomsen}{????}
\end{titlepage}
\clearpage
\fi

% Source code
\ifdefined\CMDCode
    \begin{titlepage}
    \setlength{\textwidth}{15cm}
	\noindent
	\begin{nopagebreak}
	{\samepage 
			\begin{tabular}{lr}
				\parbox{0.5\textwidth}{\raisebox{11mm}
					{\includegraphics[height=3.0cm]{Images/aauLogo.pdf}}
				} &
				\parbox{0.5\textwidth}{
					\small
					\begin{tabular}{l}
						{\sf\small \textbf{Cassiopeia}}\\
						{\sf\small \textbf{House of Computer Science}} \\
						{\sf\small Selma Lagerlöfsvej 300} \\
						{\sf\small 9220 Aalborg Ø} \\
						{\sf\small Tlf. 9940 9940} \\
						{\sf\small \url{http://www.cs.aau.dk/}}
					\end{tabular}
				}
			\end{tabular}
			
			\noindent
			\begin{tabular}{cc}
				\parbox{7cm}{
					\begin{description}
			
						\item {\bf Title:} 
						
							\textbf{\rapportnavn}\\
			            
			  
						\item {\bf Theme:}
						
							Developing Complex Software Systems
		                \newline
						\item {\bf Project Period:}\\
			  				P6, Spring 2016
			 				%\hspace{4cm}
			 			\newline
						\item {\bf Project Group:}\\
							\gruppenummer
			  				\hspace{4cm}
			  			\newline
						\item {\bf Participants:}\\
							Maritn Haaber\\
                            Kenneth Haunstrup\\
                            Kenneth Stick\\
                            Thomas Bjelbo Thomsen\\
							\hspace{2cm}
						
						\item {\bf Supervisor:}\\
							\supervisor
						\newline

						\item {\bf Copies:} write at end of project %https://www.moodle.aau.dk/course/view.php?id=11232
						\newline
						\item {\bf Number of pages:} \numpages
						\newline
						\item {\bf Appendix:} \ref{FIRSTAPPENDIX} - \ref{LASTAPPENDIX}
						\newline
						%\item {\bf Bilagsantal og art:} 1x CD
						\item {\bf Date of Completion:} \today
					\end{description}
					\vfill
				} &
				\parbox{8cm}{
					\vspace{.15cm}
					\hfill 
					\begin{tabular}{l}
						{\bf Abstract:}\bigskip \\
						\fbox{
							\parbox{6.5cm}{\smallskip
								{\vfill{\small \input{Chapters/Initial/TitlePage/Synopsis.tex}
								\smallskip}}
							}
						}
  					\end{tabular}
  				}
			\end{tabular}
		}
	\end{nopagebreak}
\end{titlepage}
\clearpage
\newcommand{\doublesignature}[4]{
    \noindent
    \begin{minipage}[b]{0.45\textwidth}
        \centering
        \rule{\textwidth}{0.5pt}\\
        #1\\
        {\footnotesize <#2@student.aau.dk>}
    \end{minipage}
    \hfill
    \begin{minipage}[b]{0.45\textwidth}
        \centering
        \rule{\textwidth}{0.5pt}\\
        #3\\
        {\footnotesize <#4@student.aau.dk>}
        \end{minipage}
    \vspace{4\baselineskip}
% Very important whitespace

}

\begin{titlepage}
    \vspace*{\fill}
    \doublesignature{Maritn Haaber}{mhaabe13}{Kenneth Haunstrup}{??}
    \doublesignature{Kenneth Stick}{????}{Thomas Bjelbo Thomsen}{????}
\end{titlepage}
\clearpage
\fi

% Oprettelse af links inde i pdf dokumentet
\hypersetup{
    pdftitle={\rapportnavn},
    pdfauthor={\gruppen},
    pdfsubject={\rapportnavn},
    bookmarksnumbered=true,
    bookmarksopen=false,
    bookmarksopenlevel=1,
    colorlinks=false,
    pdfstartview=Fit,
    pdfpagemode=UseOutlines,
    pdfpagelayout=TwoPageRight,
    pdfborder = {0 0 0}
}

% Badness underfull og overfull
\hbadness=10001                         % Slår alle underfull badness warnings fra
\hfuzz=1000pt                           % Slår de ligegyldige overfull warnings fra

% Figurer
\newcommand{\figur}[5][0]{
		\begin{figure}[H] \centering \em %H / h!
			\includegraphics[width=#5\textwidth, angle=#1]{#2}
			\caption{#3}\label{fig:#4}
		\end{figure}
}

% Inkluder ikke i ToC, men behold section number
\newcommand{\nocontentsline}[3]{}
\newcommand{\tocless}[2]{\bgroup\let\addcontentsline=\nocontentsline#1{#2}\egroup}

% Environments
\newenvironment{itemize_small}{
\begin{itemize}
  \setlength{\itemsep}{1pt}
  \setlength{\parskip}{0pt}
  \setlength{\parsep}{0pt}
}{\end{itemize}}

\newenvironment{enumerate_small}{
\begin{enumerate}
  \setlength{\itemsep}{1pt}
  \setlength{\parskip}{0pt}
  \setlength{\parsep}{0pt}
}{\end{enumerate}}

\newenvironment{grammar_indent}{
  \setlength{\grammarindent}{13em}
  \begin{grammar}
}{\end{grammar}}

\newenvironment{italic_quote}{
  \begin{quote}\itshape
}{\end{quote}}