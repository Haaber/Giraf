% Encoding set and preamble settings
\usepackage[utf8]{inputenc}
% Settings - udkommenter for at slå fra
\newcommand*{\draft}{} % Skal der compiles som draft skal slåes fra når projekt aflevers.
\newcommand*{\CMDCode}{} % Kode kommandoer
\newcommand*{\CMDFixMe}{} % Aktiver fixme

% Sæt dokumentinformationerne
\newcommand{\rapportnavn}{Somthing about giraf}
\newcommand{\rapportsubtitle}{?????????????????????????}
\newcommand{\supervisor}{Ulrik Mathias Nyman}
\newcommand{\gruppen}{sw614f16}
\newcommand{\gruppenummer}{sw614f16}
\newcommand{\afldate}{Maj 2016}

% Ekstra kommandoer
\newcommand{\figuregroup}{}
\newcommand{\tabelgroup}{}
\newcommand{\codegroup}{}
\newcommand{\screenshotgroup}{}

% Definer forkortelser
%\let\newfootnote\footnote % Kopier footnote
%\newcommand{\fn}[2]{\newfootnote{#1: #2}}
%\renewcommand{\footnote}{\errmessage{Brug fn i stedet for footnote}}

\let\newfootnote\footnote % Kopier footnote
\newcommand{\fn}[1]{\newfootnote{#1}}
\renewcommand{\footnote}{\errmessage{Brug fn i stedet for footnote}}


%TH
%${th}^$
\newcommand\nd{\textsuperscript{nd}\xspace}
\newcommand\rd{\textsuperscript{rd}\xspace}
\newcommand\nth{\textsuperscript{th}\xspace}



% Packages
\usepackage{array}% http://ctan.org/pkg/array
\usepackage{parskip} % Creates extra spacing between paragraphs
\usepackage{tabularx} % For creating tables
\usepackage{multirow} % Tables with multiple rows
\usepackage{threeparttable} % For footnotes in tables
\usepackage{fourier} % Default font
\usepackage[english]{babel} % Report language
\usepackage{geometry} % Report margins
%\usepackage[hcentering, bindingoffset=4mm]{geometry} % Report margins
\usepackage{fancyhdr} % Pretty headers and footers
\usepackage{lastpage} % For finding lastpage
\usepackage{hyperref} % clickable references for labels in PDF
\usepackage{float} % For constructing figures properly
\usepackage{wrapfig} % allows figures to wrap text -http://ctan.org/pkg/wrapfig
\usepackage{url} % For making URL
\usepackage{amssymb,amsmath} % For making large math expressions
\usepackage{titlesec} % For changing section sizes
\usepackage{xr} % Makes it possible to cross reference in multiple files
\usepackage{amsmath} % For making matematical expressions
\usepackage{mathtools} % For making matematical expressions
\usepackage[toc,page]{appendix} % Appendix
\usepackage{enumitem} % Enumerated lists
\usepackage[boxruled,linesnumbered,noend]{algorithm2e} % Algorithms
\usepackage{graphicx} % Images
\usepackage{pdfpages} % PDF pages
\usepackage{color} % For using colors
\usepackage{soulutf8} % For highlighting
\usepackage{multicol} % For itemizing with multiple columns
\usepackage{cleveref} % Enhances LATEX's cross-referencing features
\usepackage[nounderscore]{syntax} % Syntax for Context Free Grammar
\usepackage{listings} % For writing code.
\usepackage{csquotes} % Used for making quotes
\usepackage[backend=biber, style=ieee]{biblatex} % Litterature list
\usepackage{todonotes} %adds todo list
\usepackage{blindtext} % adds some test without meanning for a todo


\usepackage{xspace}

\setlength\extrarowheight{4pt} % Extra spacing in tables between row elements

% Draft settings
\ifdefined\draft\usepackage{Preamble/Packages/currfile}\fi % Show file names
\ifdefined\draft\usepackage[inline]{showlabels}\fi % Show label names

% Insert litterature
\addbibresource{Chapters/Kilder.bib}

% Headlines for algorithms
% http://tex.stackexchange.com/questions/41012/change-the-algorithm-enumeration-in-the-algorithm2e-package
\renewcommand{\listalgorithmcfname}{Algorithms}%
\renewcommand{\algorithmcfname}{Algorithm}%
\renewcommand{\algorithmautorefname}{algorithm}%
\renewcommand{\algorithmcflinename}{line}%
\renewcommand{\procedureautorefname}{procedure}%

% Reference for listings
\crefname{lstlisting}{listing}{listings}

% https://tex.stackexchange.com/questions/129400/multiple-letters-without-spacing-in-math
\newcommand{\mli}[1]{\mathit{#1}}

% Dokument informationskommandoer
\newcommand{\currentpage}{\thepage}     % nuværende side
\newcommand{\numpages}{\pageref{LastPage}}% antal sider
\newcommand{\sidetal}{Page {\currentpage} of {\numpages}} % side informationer
\newcommand{\pagetitle}{\rightmark}     % nuværende sektion

% Report meta information
\title{\rapportnavn}
\author{\gruppen}
\date{\afldate}

% Setup af header og footer
\fancypagestyle{plain}{
    % Nulstil header og footer
    \fancyhead{}
    \fancyfoot{}

    % Setup header
    % Left / Right - Even / Odd
    \fancyhead[LE]{\rapportnavn}        % Lige sider
    \fancyhead[RO]{\pagetitle}          % Ulige sider

    % Sæt footer op
    \fancyfoot[LE]{\sidetal}            % Lige sider
    \fancyfoot[RO]{\sidetal}            % Ulige sider
    
    % Sæt kun filnavn på en draft
    \ifdefined\draft
        \fancyfoot[LO]{\currfilepath}   % Ulige sider
        \fancyfoot[RE]{\currfilepath}   % Lige sider
    \fi
}

\setlength{\headheight}{15pt}

% Sætter pagestyle til plain - header og footer
\pagestyle{plain}

% Fjerner ligegyldig whitespace - men rykker ting op så vi ikke har teksten på slut
\raggedbottom

% Fjerner punktummet efter sektion-nummer i header (2.2. XXX --> 2.2 XXX)
\renewcommand\sectionmark[1]{%
  \markright{\MakeUppercase{\textbf{\thesection}\ #1}}}

% Fjerner afstand mellem kapitel og sætter størrelsen på skriften - se evt. her: http://tex.stackexchange.com/questios/63390/how-to-decrease-spacing-before-chapter-title
\titleformat{\chapter}[display]{\normalfont\huge\bfseries}{\chaptertitlename \hspace{0.1cm} \thechapter}{20pt}{\Huge}
\titlespacing*{\chapter}{0pt}{0pt}{20pt}

% Indent til eller fra
\newcommand{\enableIndent}{\setlength\parindent{12pt}}
\newcommand{\disableIndent}{\setlength\parindent{0pt}}
\disableIndent 

% Opretter subsubsubsection og subsubsubsubsection
\newcommand{\subsubsubsection}[1]{\noindent\paragraph{#1}\mbox{}\\}
\newcommand{\subsubsubsubsection}[1]{\noindent\subparagraph{#1}\mbox{}\\}

% Skriftstørrelse på sections
\titleformat*{\section}{\LARGE\bfseries}
\titleformat*{\subsection}{\Large\bfseries}
\titleformat*{\subsubsection}{\large\bfseries}
\titleformat*{\paragraph}{\normalsize\bfseries}
\titleformat*{\subparagraph}{\normalsize\bfseries}

% Definerer antallet af overskrifter der har tal, samt antallet af overskrifter i indholdsfortegnelsen
\setcounter{secnumdepth}{2}             % Antallet af overskrifter der har et nr - dybden
\setcounter{tocdepth}{1}                % Antallet af overskrifter i indholdsfortegnelse - dybden - eks. 3.2

% Orddeling
\hyphenation{}

% Ændrer indstillingerne for caption for figurer, tabeller og kodestykker - tilføjer bla. kusiv og gør typenavnet bold (eks. Figur 1.1: bliver fed)
\usepackage[font=small,format=plain,labelfont=bf,up,textfont=it,up]{caption}

% FixMe
\ifdefined\CMDFixMe
    % FixMe
\ifdefined\draft
    \usepackage[draft,english]{fixme}    % Fixme draft
\else
    \usepackage[final,english]{fixme}    % Fixme final
\fi

% FixMe
% http://mirrors.dotsrc.org/ctan/macros/latex/contrib/fixme/fixme.pdf
\newcommand{\fix}[2]{\fxfatal{\textcolor{red}{#1} /#2}}
\newcommand{\note}[2]{\fxnote{#1 /#2}}
\newcommand{\anfix}[3]{
    \begin{anfxfatal}{\textcolor{red}{#2} /#3} 
        \textcolor{red}{#1} /#3
    \end{anfxfatal}
}

\newcommand{\notdonefix}[1]{\fxnote{\textcolor{blue}{Not done (#1)}}}

\renewcommand\fxdanishfatalname{Error}
\renewcommand\fxdanishnotename{Note}
\renewcommand{\fixme}[1]{\fix{Don't use this fixme command}}

%Slet fixme (udkommenter fixme når reporten skal aflevers.)
\let\fixme\relax

%Fix
\newcommand{\hfix}[1]{\fix{#1}{Haaber}}
\newcommand{\kfix}[1]{\fix{#1}{Kenneth}}
\newcommand{\sfix}[1]{\fix{#1}{Stick}}
\newcommand{\tfix}[1]{\fix{#1}{Thomas}}


\newcommand{\notdone}[1]{\notdonefix{#1}}
\fi

% Source code
\ifdefined\CMDCode
    % Sourcecode highlight i utf8 format
\usepackage{listingsutf8}

% Danske bogstaver i kodefiler
\lstset{literate=%
    {æ}{{\ae}}1
    {å}{{\aa}}1
    {ø}{{\o}}1
    {Æ}{{\AE}}1
    {Å}{{\AA}}1
    {Ø}{{\O}}1
}

% Source code tekster
%\renewcommand{\lstlistingname}{Code example}
%\renewcommand{\lstlistlistingname}{Code examples}
%Definer farver
\definecolor{gray95}{gray}{.95}
\definecolor{bluekeywords}{rgb}{0.13,0.13,1}
\definecolor{greencomments}{rgb}{0,0.5,0}
\definecolor{redstrings}{rgb}{0.9,0,0}

% Definer style
\lstdefinestyle{Default}
{ 
    numbers=left,
	numbersep=5pt,
	stepnumber=1,
	captionpos=b,
	keywordstyle=\color{bluekeywords},
	commentstyle=\color{greencomments},
	stringstyle=\color{redstrings},
	backgroundcolor=\color{gray95},
	frame=lrtb,
	framerule=0.5pt,
	linewidth=1.00\textwidth,
	tabsize=4,
	numberbychapter=true,
	basicstyle=\ttfamily\footnotesize, %\ttfamily
	breaklines=true,
	breakatwhitespace=true,
	showstringspaces=false,
	showspaces=false,
	showtabs=false,
	breakindent=20pt
}

\lstset{style=Default}

% Indlæs kode fra fil med style
\newcommand{\kodeprintstyle}[4]{
    \lstinputlisting[language=#4, style=Default, caption={#2}, label={#3}]{#1}
}
\newcommand{\AnsiC}[3]{
    \kodeprintstyle{#1}{#2}{#3}{C}
}
\lstdefinelanguage{CSharp} {
    language=[Sharp]C,
    morekeywords={var},
    morekeywords={from,where,orderby,descending,select}, % LINQ
    morekeywords={get,set}
}

\newcommand{\CSharp}[3]{
    \kodeprintstyle{#1}{#2}{#3}{CSharp}
}
\lstdefinelanguage{xaml} {
    morekeywords={BooleanExpression},
    alsoletter={:,,/,?},
    morestring=[b]{"},
    morecomment=[s]{&lt;!--}{--&gt;},keywordstyle=\color{forestGreen},
    morekeywords={TypeArguments,Name,Default,DisplayName,OperationName,ServiceContractName,Key,AddressUri,
CanCreateInstance, LogName, Message, MessageNumber, Expression,CorrelationHandle,Request}
}

\newcommand{\XAML}[3]{
    \kodeprintstyle{#1}{#2}{#3}{xaml}
}
\newcommand{\XML}[3]{
    \kodeprintstyle{#1}{#2}{#3}{XML}
}
\newcommand{\sql}[3]{
    \kodeprintstyle{#1}{#2}{#3}{SQL}
}
\newcommand{\html}[3]{
    \kodeprintstyle{#1}{#2}{#3}{HTML}
}
\lstdefinelanguage{Java} {
    keywords={true, false, return, null, try, catch, switch, if, while, do, else, for, instanceof, private, public},
    keywordstyle=\color{blue}\bfseries,
    ndkeywords={import, static, @Test, @Override, int, double, class},
    ndkeywordstyle=\color{violet}\bfseries,
    identifierstyle=\color{black},
    sensitive=false,
    comment=[l]{//},
    morecomment=[s]{/*}{*/},
    commentstyle=\color{gray}\ttfamily,
    stringstyle=\color{dkgreen}\ttfamily,
    morestring=[b]"
}

\newcommand{\Java}[3]{
    \kodeprintstyle{#1}{#2}{#3}{Java}
}
\colorlet{punct}{red!60!black}
\definecolor{background}{HTML}{EEEEEE}
\definecolor{delim}{RGB}{20,105,176}
\colorlet{numb}{magenta!60!black}

\lstdefinelanguage{json}{
    basicstyle=\normalfont\ttfamily,
    numbers=left,
    numberstyle=\scriptsize,
    stepnumber=1,
    numbersep=8pt,
    showstringspaces=false,
    breaklines=true,
    frame=lines,
    backgroundcolor=\color{background},
    literate=
     *{0}{{{\color{numb}0}}}{1}
      {1}{{{\color{numb}1}}}{1}
      {2}{{{\color{numb}2}}}{1}
      {3}{{{\color{numb}3}}}{1}
      {4}{{{\color{numb}4}}}{1}
      {5}{{{\color{numb}5}}}{1}
      {6}{{{\color{numb}6}}}{1}
      {7}{{{\color{numb}7}}}{1}
      {8}{{{\color{numb}8}}}{1}
      {9}{{{\color{numb}9}}}{1}
      {:}{{{\color{punct}{:}}}}{1}
      {,}{{{\color{punct}{,}}}}{1}
      {\{}{{{\color{delim}{\{}}}}{1}
      {\}}{{{\color{delim}{\}}}}}{1}
      {[}{{{\color{delim}{[}}}}{1}
      {]}{{{\color{delim}{]}}}}{1},
}

\newcommand{\json}[3]{
    \kodeprintstyle{#1}{#2}{#3}{json}
}
\lstdefinelanguage{Julia}%
  {morekeywords={abstract,break,case,catch,const,continue,do,else,elseif,%
      end,export,false,for,function,immutable,import,importall,if,in,%
      macro,module,otherwise,quote,return,switch,true,try,type,typealias,%
      using,while},%
   sensitive=true,%
   alsoother={$},%
   morecomment=[l]\#,%
   morecomment=[n]{\#=}{=\#},%
   morestring=[s]{"}{"},%
   morestring=[m]{'}{'},%
}[keywords,comments,strings]%


\newcommand{\Julia}[3]{
    \kodeprintstyle{#1}{#2}{#3}{Julia}
}
\newcommand{\MATLAB}[3]{
    \kodeprintstyle{#1}{#2}{#3}{Matlab}
}
\newcommand{\Python}[3]{
    \kodeprintstyle{#1}{#2}{#3}{Python}
}
\newcommand{\R}[3]{
    \kodeprintstyle{#1}{#2}{#3}{R}
}
\fi

% Oprettelse af links inde i pdf dokumentet
\hypersetup{
    pdftitle={\rapportnavn},
    pdfauthor={\gruppen},
    pdfsubject={\rapportnavn},
    bookmarksnumbered=true,
    bookmarksopen=false,
    bookmarksopenlevel=1,
    colorlinks=false,
    pdfstartview=Fit,
    pdfpagemode=UseOutlines,
    pdfpagelayout=TwoPageRight,
    pdfborder = {0 0 0}
}

% Badness underfull og overfull
\hbadness=10001                         % Slår alle underfull badness warnings fra
\hfuzz=1000pt                           % Slår de ligegyldige overfull warnings fra

% Figurer
\newcommand{\figur}[5][0]{
		\begin{figure}[H] \centering \em %H / h!
			\includegraphics[width=#5\textwidth, angle=#1]{#2}
			\caption{#3}\label{fig:#4}
		\end{figure}
}

% Inkluder ikke i ToC, men behold section number
\newcommand{\nocontentsline}[3]{}
\newcommand{\tocless}[2]{\bgroup\let\addcontentsline=\nocontentsline#1{#2}\egroup}

% Environments
\newenvironment{itemize_small}{
\begin{itemize}
  \setlength{\itemsep}{1pt}
  \setlength{\parskip}{0pt}
  \setlength{\parsep}{0pt}
}{\end{itemize}}

\newenvironment{enumerate_small}{
\begin{enumerate}
  \setlength{\itemsep}{1pt}
  \setlength{\parskip}{0pt}
  \setlength{\parsep}{0pt}
}{\end{enumerate}}

\newenvironment{grammar_indent}{
  \setlength{\grammarindent}{13em}
  \begin{grammar}
}{\end{grammar}}

\newenvironment{italic_quote}{
  \begin{quote}\itshape
}{\end{quote}}