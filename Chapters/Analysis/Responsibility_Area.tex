\section{Responsibility area}

\begin{description}
    \item[Product Owner:] The group that have the communication with the costumer, and are in charged of the backlog. 
    \item[Scrum Master:] A group in GIRAF that enforce the development method.
    \item[Server \& Database:] In GIRAF there are two groups that are in charge of making the server and database working.
    \item[Security:] A group that has has a overview of the security that have to be in the project.
    \item[Unit and Integration Tests:] A group there is in charged of how to do unit testing. 
    \item[Google Analytics:] A group that are in charged of Google bug reporting and updating application on Google Play.
    \item[Graphics and sound:] A group that are in charged of how graphics and sound for GIRAF should look.
    \item[Documentation \& Wiki:] A group that know what to do with JavaDoc and how to write the guides.
    \item[Usability Tests:] A group that are in charged if making the usability test layout.
    \item[Social event planners:] A group that has the job of planning social events for the entire multi-project group.
\end{description}

In this project this group has chosen to have to responsibility area which will be described

\subsection{Product Owner}
In this context of GIRAF there is one top level product owner, and our group has taken on that responsibility. Some of the work that is involved with this responsibility is to talk to the costumer, at this meeting we show what has happened over the last sprint and talk about new features they would want in GIRAF and makes these wishers into a user story which then are put on the backlog, the backlog is then prioritised by us the way the backlog is prioritised is after the focus of the sprint. With the black log prioritised we can now make it public and all groups can now look at it and tasks can be assigned to groups.


\subsection{Security}